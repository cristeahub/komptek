%%%%%%%%%%%%%%%%%%%%%%%%%%%%%%%%%%%%%%%%%
% Short Sectioned Assignment
% LaTeX Template
% Version 1.0 (5/5/12)
%
% This template has been downloaded from:
% http://www.LaTeXTemplates.com
%
% Original author:
% Frits Wenneker (http://www.howtotex.com)
%
% License:
% CC BY-NC-SA 3.0 (http://creativecommons.org/licenses/by-nc-sa/3.0/)
%
%%%%%%%%%%%%%%%%%%%%%%%%%%%%%%%%%%%%%%%%%

%----------------------------------------------------------------------------------------
% PACKAGES AND OTHER DOCUMENT CONFIGURATIONS
%----------------------------------------------------------------------------------------

\documentclass[paper=a4, fontsize=11pt]{scrartcl} % A4 paper and 11pt font size

\usepackage[T1]{fontenc} % Use 8-bit encoding that has 256 glyphs
\usepackage{fourier} % Use the Adobe Utopia font for the document - comment this line to return to the LaTeX default
\usepackage[norsk]{babel} 
\usepackage[utf8]{inputenc}
\usepackage{amsmath,amsfonts,amsthm} % Math packages
\usepackage{tikz} % drawing package

\usepackage{sectsty} % Allows customizing section commands
\allsectionsfont{\centering \normalfont\scshape} % Make all sections centered, the default font and small caps

\usepackage{fancyhdr} % Custom headers and footers
\pagestyle{fancyplain} % Makes all pages in the document conform to the custom headers and footers
\fancyhead{} % No page header - if you want one, create it in the same way as the footers below
\fancyfoot[L]{} % Empty left footer
\fancyfoot[C]{} % Empty center footer
\fancyfoot[R]{\thepage} % Page numbering for right footer
\renewcommand{\headrulewidth}{0pt} % Remove header underlines
\renewcommand{\footrulewidth}{0pt} % Remove footer underlines
\setlength{\headheight}{13.6pt} % Customize the height of the header

\numberwithin{equation}{section} % Number equations within sections (i.e. 1.1, 1.2, 2.1, 2.2 instead of 1, 2, 3, 4)
\numberwithin{figure}{section} % Number figures within sections (i.e. 1.1, 1.2, 2.1, 2.2 instead of 1, 2, 3, 4)
\numberwithin{table}{section} % Number tables within sections (i.e. 1.1, 1.2, 2.1, 2.2 instead of 1, 2, 3, 4)

\setlength\parindent{0pt} % Removes all indentation from paragraphs - comment this line for an assignment with lots of text

%---- Listings --------------
\usepackage{color}
\definecolor{light-gray}{gray}{0.95}
\usepackage{listings}

\lstnewenvironment{code}[1][]%
{\minipage{\linewidth}
\lstset{ %
language={[x86masm]Assembler},  % choose the language of the code
basicstyle=\footnotesize,       % the size of the fonts that are used for the code
numbers=left,                   % where to put the line-numbers
numberstyle=\footnotesize,      % the size of the fonts that are used for the line-numbers
stepnumber=1,                   % the step between two line-numbers. If it is 1 each line will be numbered
resetmargins=true,              % reset line numbers
numbersep=5pt,                  % how far the line-numbers are from the code
backgroundcolor=\color{white},  % choose the background color. You must add \usepackage{color}
showspaces=false,               % show spaces adding particular underscores
showstringspaces=false,         % underline spaces within strings
showtabs=false,                 % show tabs within strings adding particular underscores
frame=single,                   % adds a frame around the code
tabsize=2,                      % sets default tabsize to 2 spaces
captionpos=b,                   % sets the caption-position to bottom
breaklines=true,                % sets automatic line breaking
breakatwhitespace=false,        % sets if automatic breaks should only happen at whitespace
escapeinside={\%*}{*)},         % if you want to add a comment within your code
morekeywords={
    orr, ldr, bne, subs
    },                          % additional keywords to expand the asm language
#1
}}%
{\endminipage}

%----------------------------------------------------------------------------------------
% TITLE SECTION
%----------------------------------------------------------------------------------------

\newcommand{\horrule}[1]{\rule{\linewidth}{#1}} % Create horizontal rule command with 1 argument of height

\title{ 
\normalfont \normalsize 
\textsc{TDT4205 Compilers, NTNU} \\ [25pt] % Your university, school and/or department name(s)
\horrule{0.5pt} \\[0.4cm] % Thin top horizontal rule
\huge Problem Set 5 \\ % The assignment title
\horrule{2pt} \\[0.5cm] % Thick bottom horizontal rule
}

\author{Christoffer Tønnessen} % Your name

\date{\normalsize7. march 2014} % Today's date or a custom date

\begin{document}

\maketitle % Print the title

%----------------------------------------------------------------------------------------
% PROBLEM 1
%----------------------------------------------------------------------------------------

\section{Type checking}
%----------------------------------------------------------------------------------------
There are some typecheking which is not done in our compiler.
This causes some programs to have errors as default output, even though they are technically correct.

One check would be to type check class fields.

Another check that is not done is return statements.
In this instance one would need to check if the variable returned has the correct type of the function.
This is very useful to do, and most modern languages does this.
This could be fixed with some type conversion, which will be mentioned more.

Expressions in if-statements have no typecheck.
To ensure that one checks two elements in a matter that makes sense, typechecking would be a good thing.

With expressions in if-statements as well as the case with the return statement, one could introduce type conversion.
Type conversion checks both types and sees what probably should be and changes one of the types accordingly.
This is done in a veraity of different programming languages and has it's goods and bads.
The good thing is that the program gets less verbose and it can be easier for the programmer.
The bad part is that sometimes it leads to unexpected behavior.

%----------------------------------------------------------------------------------------
% PROBLEM 2
%----------------------------------------------------------------------------------------
\newpage
\section{Assembly programming}

At position 1 the stack frame will look as following:

\begin{figure}[ht!]
    \begin{center}
    \begin{tabular}{|l|l|}
    \hline
    Memory    & Variable       \\
    \hline
    main\_fp  & saved fp       \\
    \hline
    main\_fp-4 & a\_main        \\
    \hline
    main\_fp-8 & b\_main        \\
    f\_fp+8   & a\_f           \\
    \hline
    f\_fp+4   & return address \\
    \hline
    f\_fp     & saved fp       \\
    \hline
    f\_fp-4   & b\_f           \\
    \hline
    f\_fp-8   & c\_f           \\
    \hline
    f\_fp-12  & d\_f           \\
    \hline
    g\_fp+16  & a\_g           \\
    \hline
    g\_fp+12  & b\_g           \\
    \hline
    g\_fp+8   & c\_g           \\
    \hline
    g\_fp+4   & return address \\
    \hline
    g\_fp     & saved fp       \\
    \hline
    g\_fp-4   & e\_g           \\
    \hline
    g\_fp-8   & d\_g           \\
    \hline
    \end{tabular}
    \end{center}
\end{figure}

%----------------------------------------------------------------------------------------
% PROBLEM 3
%----------------------------------------------------------------------------------------
\section{Assembly Programming}

Here is the assembly code for the functions.
\begin{figure}[ht!]
\begin{code}
_f:
    push {lr}
    push {fp}

    mov fp, sp

    ldr r2, [fp, #12]
    ldr r1, [fp, #8]
    str r1, [fp, #-4]

    push {[fp, #12]}
    push {[fp, #-4]}

    bl _g

    pop % pop old parameters
    pop
    pop {fp}
    pop {pc}

_g:
    push {lr}
    push {fp}

    mov fp, sp

    ldr r0, [fp, #8]

    pop {fp}
    pop {pc}
\end{code}
\end{figure}
%----------------------------------------------------------------------------------------
\end{document}
